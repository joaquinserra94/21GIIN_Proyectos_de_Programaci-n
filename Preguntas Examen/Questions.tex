// Questions


1. **¿Qué es la herencia en POO?**
   - La herencia es un mecanismo que permite crear una nueva clase a partir de una clase existente. La nueva clase, llamada clase derivada o hija, hereda atributos y métodos de la clase base o padre.

2. **¿Cuál es la diferencia entre herencia simple y herencia múltiple?**
   - La herencia simple es cuando una clase hereda de una sola clase base. La herencia múltiple es cuando una clase puede heredar de más de una clase base. En Java, solo se permite la herencia simple, mientras que en otros lenguajes como C++ se permite la herencia múltiple.

3. **¿Qué es la sobrescritura de métodos?**
   - La sobrescritura de métodos ocurre cuando una clase derivada proporciona una implementación específica de un método que ya está definido en su clase base. Esto permite que la clase derivada tenga su propio comportamiento para ese método.

4. **¿Qué es el polimorfismo y cómo se relaciona con la herencia?**
   - El polimorfismo es la capacidad de un objeto de tomar muchas formas. En el contexto de la herencia, permite que una referencia de la clase base apunte a objetos de la clase derivada, y se llame al método sobrescrito de la clase derivada.

5. **¿Qué es el uso de `super` en Java?**
   - La palabra clave `super` se utiliza en Java para referirse a la clase base inmediata. Se puede usar para llamar al constructor de la clase base o para acceder a métodos y atributos de la clase base que han sido sobrescritos en la clase derivada.

6. **¿Qué es la herencia en cascada?**
   - La herencia en cascada se refiere a una cadena de herencia donde una clase hereda de otra, que a su vez hereda de otra, y así sucesivamente. Esto crea una jerarquía de clases.

7. **¿Cuáles son las ventajas de usar herencia?**
   - Reutilización de código: Permite reutilizar el código de la clase base en las clases derivadas.
   - Mantenimiento: Facilita el mantenimiento del código, ya que los cambios en la clase base se reflejan en las clases derivadas.
   - Extensibilidad: Permite extender la funcionalidad de las clases existentes sin modificar su código.

8. **¿Cuáles son las desventajas de usar herencia?**
   - Complejidad: Puede aumentar la complejidad del diseño y la comprensión del código.
   - Acoplamiento: Las clases derivadas están estrechamente acopladas a la clase base, lo que puede dificultar los cambios en la clase base sin afectar a las clases derivadas.
